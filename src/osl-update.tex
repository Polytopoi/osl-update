\documentclass[11pt]{article}

\usepackage{amsfonts}
\usepackage{amsmath}
\usepackage[total={7in,9in}]{geometry}
\usepackage{amsthm}
\usepackage{graphicx}
\usepackage{lmodern}
\usepackage{hyperref}

\title{Arithmetization using application specific trace types}
\author{ \\ Morgan Thomas \\ Casper Association \\ morgan@casper.network }


\begin{document}

\maketitle

\begin{abstract}
	This note provides an update on the Open Specification Language (OSL) circuit compiler. OSL is a language based on predicate logic which is amenable to compilation to arithmetic constraint systems for use in constructing (zk-)SNARKs. This system provides an alternative to universal zk-VMs and low level \emph{ad hoc}\/ constructions of arithmetic constraint systems, which is potentially more efficient than universal zk-VMs but more cost effective as a development approach than low level \emph{ad hoc}\/ constructions.
\end{abstract}


Arithmetization is the process of expressing a relation as an arithmetical constraint system suitable for proving membership in the relation using (zk-)SNARKs. There are two kinds of approaches to arithmetization: universal and application-specific. Universal arithmetization uses constraint systems which can capture any NP relation up to some complexity bounds. Application-specific arithmetization uses \emph{ad hoc}\/ constraint systems which each capture one NP relation (again, up to some complexity bounds).

Universal arithmetization typically works by expressing the semantics of a virtual machine architecture as an arithmetical constraint system, yielding a so-called zero knowledge virtual machine (zk-VM). Application specific arithmetization typically works by writing a constraint system more or less directly within some cryptographic application development framework, or else by using a compiler which transforms a relation expressed in some language into an arithmetical constraint system.

In zero knowledge proving for a relation $R$, one wants to show, for some public $x$, that there exists some (private) $w$ such that $R(x, w)$. Oftentimes, one models the relation to be arithmetized as a function; in general, for a function $f$, one wants to show, for some public $x$ and $y$, that there exists some (private) $w$ such that $f(x, w) = y$. The function formulation is a special case of the relation formulation where one lets $R = \{((x, y), w) : f(x, w) = y\}$. One can also go from the relation formulation to the function formulation, given an algorithm for checking membership in the relation $R$, that is, a function $f$ such that $f(x, w) = 1$ iff $R(x, w)$.

The goal of this research is to develop tools for creating zk-SNARKs for a relation $R$ without being concerned about how the relation $R$ gets checked. Such tools would allow for working at a higher level of abstraction, separating the concern of \emph{what}\/ is to be proven from \emph{how}\/ it is to be proven. zk-VM approaches require as input a program and inputs to the program, thus requiring the end developer to specify a way to check the statement using a program execution.

As an alternative, this research would allow end developers to specify a relation $R$ by a formula $\Phi(x, w)$ in a version of predicate logic, using a circuit compiler to generate an arithmetical constraint system for the relation $R$, and an argument translator which takes inputs $(x, w)$ such that $R(x, w)$ and turns them into inputs satisfying the arithmetical constraint system.

This research began with the support of Orbis Labs and continues with the support of Casper Association. Prior art in this line of research \cite{sigma11,osl-paper,osl-poly-bounds} defined a version of predicate logic which is strong enough to express NP relations which are practical to arithmetize, and also some methods for arithmetizing formulas in that language on top of PLONKish arithmetization using Halo 2. \cite{halo2-book,halo2-github} The language in question is called OSL, which originally stood for Orbis Specification Language and now stands for Open Specification Language (a backronym). Implementation efforts \cite{osl-github} are ongoing and open source. As of this writing, these efforts are closely approaching a point where generating zero knowledge proofs will be possible. The current paper provides an update on the research and development efforts, which have come up with more practical methods of arithmetization than what are described in the prior literature.

OSL is a layer of abstraction over $\Sigma^1_1$ formulas. $\Sigma^1_1$ formulas are a sublanguage of second order arithmetic \cite{simpson} extended with $\text{ind}_<$ and max built-in functions, defined as follows:

\begin{equation}
	\text{ind}_<(x, y) = \begin{cases}
		1 & \text{if}\ x < y, \\
		0 & \text{otherwise}.
	\end{cases}
\end{equation}

\begin{equation}
	\text{max}(x, y) = \begin{cases}
		x & \text{if} y \leq x, \\
		y & \text{otherwise}.
	\end{cases}
\end{equation}

$\Sigma^1_1$ formulas allow for first-order quantifiers with polynomial bounds, and second-order existential quantifiers with polynomial bounds. See \cite{sigma11-poly-bounds} for more details and a formal definition of $\Sigma^1_1$ formulas. OSL adds rudiments of type theory, including basic functors like list and map type constructors, all of which can be straightforwardly translated into $\Sigma^1_1$ formulas.

There is a natural connection between $\Sigma^1_1$ formulas and computation, in that sequential models of computation are typically natural to express as $\Sigma^1_1$ formulas. If a function $f$ is defined within a sequential model of computation, then $f(x, w) = y$ means in other words that there is a sequence of states $s_0, ..., s_n$ such that $s_0$ is the appropriate initial state for inputs $(x, w)$, $s_n$ is a final state where the program has halted and indicated an output of $y$, and for all $i \in [0, n)$, the step transition function of the model of computation maps $s_i$ to $s_{i+1}$. The same can be expressed as a $\Sigma^1_1$ formula as long as the step transition function is a first-order relation (i.e., it can be expressed using only first-order polynomially bounded quantifiers).

There is also a natural connection between $\Sigma^1_1$ formulas and arithmetical constraint systems. The semantics of arithmetical constraint systems are typically natural to express as $\Sigma^1_1$ formulas, and it is also very feasible to express a $\Sigma^1_1$ formula as an arithmetical constraint system. The free variables in the formula (the ones not bound by quantifiers) correspond to public inputs, and the variables bound by existential quantifiers correspond to private inputs. In a useful variant on this approach, we consider $\Sigma^1_1$ formulas with no free variables, and we add a so-called instance quantifier, $\lambda$, where $\lambda x.\ \Phi(x)$ has the same semantics as $\Phi(x)$ (where $x$ is free in $\Phi(x)$). What is useful about the instance quantifier concept is that we can use it to apply a bound to an instance variable, as in $\lambda x < \alpha.\ \Phi(x)$.

It is in general not computationally feasible to compute whether a $\Sigma^1_1$ formula is true or false. However, it is in general computationally feasible to check whether a first-order formula is satisfied or not on given inputs. Given satisfying values for the instance or second-order existential quantifiers in a $\Sigma^1_1$ formula, it is in general computationally feasible to check that the formula is true, since this reduces to the problem of determining that a first-order formula is true after plugging in the given values for the second-order variables.

Prior research \cite{sigma11,sigma11-poly-bounds} described methods for arithmetizing $\Sigma^1_1$ formulas which took advantage of the similarities between $\Sigma^1_1$ formulas and arithmetical constraint systems. These methods, however, were not feasible to apply in practice. They tended to result in a blow-up in the degrees and numbers of terms in the resulting polynomial constraints, to a point that it was often not feasible to compute these polynomials, let alone generate zk-SNARKs based on them.

The current version of the OSL compiler \cite{osl-github} demonstrates a feasible method of compiling $\Sigma^1_1$ formulas to arithmetical constraint systems. This method is based on application specific trace types.

A ``trace'' describes the steps of a computational process. For a sequential model of computation, a trace is equivalent to a series of states $s_0, ..., s_n$, with $s_0$ being the initial state and $s_n$ the final state, such that for all $i \in [0,n)$, the step transition relation for the model of computation relates $s_i$ to $s_{i+1}$. More generally, a trace need not have a sequential structure, but can have a tree structure.

For examples of non-sequential traces, we can consider traces for an arithmetic circuit. An arithmetic circuit, let's say, is a directed acyclic graph, where each non-initial node (each node with an ancestor) has exactly two ancestors and is labeled with either $+$ or $\times$, and there is exactly one final node (a node which is not an ancestor). A trace for an arithmetic circuit is a function from the set of nodes to the set of field elements.

A ``trace type'' is an object which defines what would be a valid trace for that trace type. For example, a trace type can define valid traces for a specific arithmetic circuit, or valid traces for a specific model of computation. The reason for defining a general trace type abstraction is in order to define application specific trace types for arithmetizing relations using application specific constraint systems. The overall approach is to define a compiler which compiles relations defined in OSL into $\Sigma^1_1$ formulas, which are then compiled into application specific trace types, which are then compiled into arithmetical constraint systems.

It would of course be possible to use a universal trace type to arithmetize all NP relations (up to some complexity bound). This is the approach of universal zk-VMs. The hypothesis driving this research into application specific trace types is that given specific knowledge of the relation of interest, one can automatically come up with a trace type which more efficiently arithmetizes the relation of interest.

Overall, the OSL compiler in its current form has the following stages, from first to last:

\begin{enumerate}
	\item Tokenizing turns the textual representation of OSL code into a sequence of lexical tokens.
	\item Parsing turns the tokenized OSL code into an abstract syntax tree (AST).
	\item Semantic analysis turns the AST of the OSL code into a valid context, which maps names to their definitions, ensuring that these definitions are interpretable within the context and the semantic rules of OSL.
	\item The valid OSL from semantic analysis gets compiled into a $\Sigma^1_1$ formula. In this stage, the $\Sigma^1_1$ formula uses de Bruijn indices as variable names.
	\item The de Bruijn indices in the $\Sigma^1_1$ formula get replaced with gensyms, which are globally unique generated names. Formulas using de Bruijn indices are relatively easy to generate, because unintended variable capture is not much of a concern; there is no need to carry state to remember which variable names have been used during the generation process. On the other hand, formulas using gensyms are relatively easy to manipulate. That is why this stage is useful.
	\item The $\Sigma^1_1$ formula gets converted into a prenex normal form, which brings all quantifiers to the front of the formula.
	\item The prenex normal form gets converted into a strong prenex normal form, which brings all instance quantifiers to the front of the formula, followed by all existential quantifiers, followed by all universal quantifiers.
	\item The strong prenex normal form gets compiled into a logic circuit. A logic circuit is a structure similar to a Halo 2 circuit, with some key differences, as follows. Logic circuits do not have lookup arguments, but instead just gate constraints and equality constraints. Instead of taking the form $p = 0$, for $p$ a polynomial, gate constraints are quantifier-free logic formulas which can assert equalities and inequalities over terms involving variables, constants, addition, multiplication, the max and $\text{ind}_<$ functions, and lookups.

		Variables in this context are the same as they are in Halo 2 gate constraints: they are of the form $x_{i,j}$, where $i$ is an absolute column index and $j$ is a relative row index (an offset from the current row). Lookups in this context are of the form $c_{n+1}((t_1, c_1), ..., (t_n, c_n))$ where each $t_i$ is a term and each $c_i$ is a column ($i$ being an absolute column index). The output of a lookup term at a row $r$ is the value of column $c_{n+1}$ in the unique row $r'$ such that the vector of column values $(c_1, ..., c_n)$ at row $r'$ is equal to the vector of term values $(t_1, ..., t_n)$ at row $r$. Lookups are effectively the ``missing link'' between function calls and lookup arguments, having some characteritics of both function calls and lookup arguments, being the result of compiling function calls and being compiled into lookup arguments in Halo 2 circuits.

		This notion of logic circuits is similar to the notion of logic circuits found in \cite{sigma11-poly-bounds}, with the difference being that terms related by equalities and inequalities are as just described instead of being plain polynomials as in \cite{sigma11-poly-bounds}.

	\item The logic circuit gets compiled into a trace type.
	\item The trace type gets compiled into a Halo 2 circuit.
\end{enumerate}

For practical purposes, one needs not only a compiler to turn OSL code into Halo 2 circuits, but also a compiler to turn inputs satisfying a predicate defined in OSL into inputs satisfying the resulting Halo 2 circuit. The latter compiler is called the ``argument compiler.'' The OSL compiler only needs to be applied once per relation to be arithmetized, and the argument compiler needs to be applied once per instance to be proven.

As of this writing, the OSL compiler and the argument compiler exist and pass tests, but they use a notion of circuits which is not exactly compatible with Halo 2. The disconnect is that the notion of circuits targeted in the OSL compiler features lookup arguments gated by polynomial expressions, such that the lookup arguments only apply at rows where the gate expression is zero, whereas Halo 2 only supports lookup arguments gated by fixed selector columns. There is nothing that prevents this gap from being bridged efficiently, and work to bridge the gap is underway as of this writing.

In the previously published constructions of Halo 2 circuits from logic circuits, \cite{sigma11,sigma11-poly-bounds}, each row of the Halo 2 circuit corresponds to one row of the source logic circuit. Because the constraints in a logic circuit can imply a lot of complex constraints on one row, the complexity of the polynomials in the resulting Halo 2 circuits tends to be exceedingly high in the previously published constructions. The solution to this problem in the trace type based compilation pipeline is to let one row in the logic circuit correspond to many rows in the resulting Halo 2 circuit. In this newer construction, each row of the Halo 2 circuit verifies the evaluation of one subexpression at one row of the logic circuit. A subexpression can be a term, a (sub)formula of a gate constraint, or an assertion that a gate constraint formula is true.

A ``trace'' is a value which can satisfy or fail to satisfy a trace type. A trace consists of a statement, a witness, and a map from cases to maps from subexpression ids to subexpression traces. A subexpression trace consists of the value of the subexpression (a field element), the step type id of the subexpression, and the associated advice values (a map from column indexes to field elements). Each row of a logic circuit input matrix corresponds to one case of the corresponding trace. Each information bearing row of the resulting Halo 2 circuit input matrix corresponds to one subexpression trace of one case.

Similar to a Halo 2 circuit, a trace type defines a map from column indices to their column types (fixed, advice, or instance). It defines a set of equality constrainable column indices. It defines equality constraints. Each equality constraint is isomorphic to a set of cell references (which are isomorphic to pairs of column indices and (absolute) row indices). A trace type defines the values at each \emph{case}\/ of each fixed column.

A trace type defines a map from step type ids to step type definitions. A step type definition essentially defines a set of polynomial gate constraints, a set of Halo 2 lookup arguments, and some fixed values. As with the fixed values associated with the trace type overall, the fixed values associated with a step type define one value per \emph{case}, i.e. one value per row of the source logic circuit, as opposed to one value per row of the output Halo 2 circuit. This means that all rows associated with a given case have the same fixed values, as far as those fixed values defined by the step type go.

A trace type defines a set of subexpression ids and a set of subexpression links. Each subexpression link is of the form $\vec{i} \mapsto^t o$, where $o$ is a subexpression id (the output subexpression), $\vec{i}$ is a vector of subexpression ids (the input subexpressions), and $t$ is a step type id.

It is legal for there to be different subexpression links with the same output; this entails that that subexpression's value can validly be determined based on more than one step type and/or more than one input vector. So that each output subexpression value is uniquely defined by the logic circuit input matrix, it should be that in any case, all links that can be instantiated for an output $o$ result in the same value of $o$.

The motivating example of having multiple links with the same output is short circuiting evaluation. For example, if $x = 0$, then $x \times y = 0$. So we can have three links outputting $x \times y$, one of which takes $x$ as input and works when $x = 0$ and the output is 0, one of which takes $y$ as input and works when $y = 0$ and the output is 0, and one of which takes $x$ and $y$ as input and works for any values of $x$ and $y$.

A trace type defines a subset of the set of subexpression ids which are the ``result subexpression ids.'' For a trace to satisfy the trace type, each of its result subexpression ids must be true (i.e., have value 1) for each case.

A trace type defines its number of cases (equal to the number of rows of the source logic circuit) and its number of rows. The number of rows is less than or equal to $c \times |S|$, with $c$ the number of cases, and $S$ the set of subexpression ids. In practice, in the current version of the compiler, the number of rows is always equal to $c \times |S|$, but in future versions, the number of rows could be less in some cases, as an optimization, whenever it can happen that not all subexpressions need to be instantiated for all cases.

Finally, a trace type defines various column indices:
\begin{itemize}
	\item It defines a case number column index, which should be assigned, for each row, the case number that row applies to.
	\item It defines a step indicator column index, which should be assigned one at each row representing a subexpression, and zero at all other rows.
	\item It defines a step type id selection vector, which is a vector of column indices, one per step type. At each row with a step indicator value of one, the step type id selection vector should contain exactly one column index assigned the value one, with the rest being assigned zero. At rows with a step indicator value of zero, all of the step type selection vector columns can be assigned zero. The step type selection vector values indicate which step type applies to each row. 
	\item It defines a vector of input column indices, of length equal to the number of inputs to a step type, which get assigned the values of the input subexpressions to the subexpression related to each row which is used to represent a step. Note that all step types have the same number of inputs, and for step types where some of these inputs are not needed, some of these inputs are fed by the ``void'' step type, which always outputs zero and is linked to itself for all of its own inputs.
	\item It defines an output column index, which gets assigned the value of the subexpression represented by the row at each row representing a subexpression.
\end{itemize}

TODO: describe trace types

TODO: describe compiler stages

TODO: future directions: finishing, optimization, verification


\clearpage

\begin{thebibliography}{6}
	
	\bibitem{sigma11} Morgan Thomas, Orbis Labs. \textit{Arithmetization of $\Sigma^1_1$ relations in Halo 2.} Cryptology ePrint Archive, Report 2022/777. \url{https://eprint.iacr.org/2022/777}

	\bibitem{osl-github} Casper Assocation, Orbis Labs. \textit{Open Specification Language.} 2022--2023. \url{https://github.com/Polytopoi/osl}

	\bibitem{osl-paper} Morgan Thomas, Orbis Labs. \textit{Orbis Specification Language: a type theory for zk-SNARK programming.} Cryptology ePrint Archive, Report 2022/1003. \url{https://eprint.iacr.org/2022/1003}

	\bibitem{sigma11-poly-bounds} Anthony Hart, Morgan Thomas, Orbis Labs. \textit{Arithmetization of $\Sigma^1_1$ relations with polynomial bounds in Halo 2.} Cryptology ePrint Archive, Report 2022/1003. \url{https://eprint.iacr.org/2022/1005}

	\bibitem{halo2-book} The Electric Coin Company. \textit{The halo2 Book.} 2021. \url{https://zcash.github.io/halo2/index.html}

	\bibitem{halo2-github} The Electric Coin Company. \textit{halo2.} 2022. \url{https://github.com/zcash/halo2}

	\bibitem{simpson} Stephen G. Simpson. \textit{Subsystems of Second Order Arithmetic.} Cambridge University Press, 2009. ISBN 1139478915, 9781139478915.


\end{thebibliography}

\end{document}
