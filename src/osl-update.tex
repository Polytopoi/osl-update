\documentclass[11pt]{article}

\usepackage{amsfonts}
\usepackage{amsmath}
\usepackage[total={7in,9in}]{geometry}
\usepackage{amsthm}
\usepackage{graphicx}
\usepackage{lmodern}
\usepackage{hyperref}

\title{Arithmetization using application specific trace types}
\author{ \\ Morgan Thomas \\ Casper Association \\ morgan@casper.network }


\begin{document}

\maketitle

\begin{abstract}
	TODO
\end{abstract}


Arithmetization is the process of expressing a relation as an arithmetical constraint system suitable for proving membership in the relation using SNARKs.


\clearpage

\begin{thebibliography}{6}
	
	\bibitem{sigma11} Morgan Thomas, Orbis Labs. \textit{Arithmetization of $\Sigma^1_1$ relations in Halo 2.} Cryptology ePrint Archive, Report 2022/777. \url{https://eprint.iacr.org/2022/777}

	\bibitem{osl-github} Casper Assocation, Orbis Labs. \textit{Open Specification Language.} 2022--2023. \url{https://github.com/Polytopoi/osl}

	\bibitem{osl-paper} Morgan Thomas, Orbis Labs. \textit{Orbis Specification Language: a type theory for zk-SNARK programming.} Cryptology ePrint Archive, Report 2022/1003. \url{https://eprint.iacr.org/2022/1003}

	\bibitem{osl-poly-bounds} Anthony Hart, Morgan Thomas, Orbis Labs. \textit{Arithmetization of $\Sigma^1_1$ relations with polynomial bounds in Halo 2.} Cryptology ePrint Archive, Report 2022/1003. \url{https://eprint.iacr.org/2022/1005}

	\bibitem{halo2-book} The Electric Coin Company. \textit{The halo2 Book.} 2021. \url{https://zcash.github.io/halo2/index.html}

	\bibitem{halo2-github} The Electric Coin Company. \textit{halo2.} 2022. \url{https://github.com/zcash/halo2}


\end{thebibliography}

\end{document}
